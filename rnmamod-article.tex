% !TeX root = RJwrapper.tex
\title{rnmamod: An R Package for Conducting Bayesian Network Meta-analysis with Missing Participants}
\author{by Loukia M. Spineli, Chrysostomos Kalyvas, and Katerina Papadimitropoulou}

\maketitle

\abstract{%
The development of several R packages for conducting network meta-analysis has enhanced the popularity of this evidence synthesis tool. The available R packages facilitate the implementation of most models to conduct and evaluate network meta-analysis and provide the necessary results, conforming to the PRISMA-NMA statement. The rnmamod package is a novel contribution to conducting aggregate network meta-analysis using Bayesian methods, as it allows addressing missing participants properly in all models, even if a handful of the included studies report this information. Importantly, rnmamod is the first R package to offer a rich, user-friendly visualisation toolkit that turns a ``parameter-dense'' output of network meta-analysis into several comprehensive graphs. Furthermore, the package functions on various models allow processing their output to create visualisations tailored to the user preferences. Therefore, rnmamod aids the thorough appraisal and interpretation of the results, the cross-comparison of different models and the manuscript preparation for journal submission.
}

\hypertarget{introduction}{%
\section{Introduction}\label{introduction}}

Evidence-based medicine is the backbone of informed decisions for the benefit of
the patients, stemming from a meticulous and judicious use of the available evidence,
while taking into account also the clinical experience and patient values (Sackett et al. 1996).
However, the medical community is faced daily with several intervention options and
dosages, challenging the optimal practice of evidence-based medicine (Lee 2022).
Systematic reviews with pairwise meta-analysis summarise the evidence of pairs of
interventions, providing fragmented evidence that does not serve the clinical needs.
Moreover, evidence in the comparability of different interventions at the trial
level is also fragmented, as it is not feasible to compare all intervention options
for a condition in one trial. These limitations led to the development and later
establishment of network meta-analysis (NMA), also known as multiple treatment
comparison, a new generation evidence synthesis tool (Salanti 2012). Network
meta-analysis is an extension of pairwise meta-analysis for collecting all relevant
pieces of evidence for a specific condition, patient population, and intervention
options to provide coherent evidence for all possible intervention comparisons,
and allow ordering the investigated interventions from the best to worst option
for a specific outcome (Caldwell 2014). Indirect evidence (obtained from different
sets of trials sharing a common comparator) plays a central role in the development
and prominence of NMA.

Since the introduction of indirect evidence and early development of the relevant
methodology (Higgins and Whitehead 1996; Bucher et al. 1997), the NMA framework has undergone substantial
progress conceptually and methodologically. The fast-pace publications of relevant
methodological articles and systematic reviews with NMA attest to the increasing
popularity of NMA in the wide medical and evidence synthesis community
(Efthimiou et al. 2016; Petropoulou et al. 2017). Needless to say that the availability of statistical
analysis software is the driving force to the advances and wide dissemination of NMA.
A review of the methodology and software for NMA (Efthimiou et al. 2016) listed several statistical
software tools used to promote NMA, with the \textbf{R} software (R Core Team 2022) being the most
popular to develop and compare methods for NMA, followed by \textbf{Stata} (StataCorp 2021) and
\textbf{SAS software} (SAS Institute 2020).

In the last decade, there has been a raise in the R packages for NMA with various
functionalities (Dewey and Viechtbauer 2022). These packages can be categorised by, among
others:

\begin{itemize}
\tightlist
\item
  \emph{the analysis framework}: \textbf{frequentist} (\CRANpkg{netmeta} (Rücker et al. 2022), and
  \CRANpkg{NMAoutlier} (Petropoulou et al. 2027)), \textbf{Bayesian} (\CRANpkg{bnma} (Seo and Schmid 2022),
  \CRANpkg{gemtc} (van Valkenhoef and Kuiper 2021), \CRANpkg{metapack} (Lim et al. 2022), \CRANpkg{multinma} (David M. Phillippo 2022),
  \CRANpkg{NMADiagT} (Lu et al. 2020), \CRANpkg{nmaINLA} (Guenhan 2021), and \CRANpkg{pcnetmeta}
  (Lin et al. 2017)), or \textbf{both} (\CRANpkg{nmaplateplot} (Z. Wang et al. 2021), \CRANpkg{nmarank}
  (Nikolakopoulou, Schwarzer, and Papakonstantinou 2021) which is mainly frequentist-driven but can be easily applied to Bayesian
  results (Papakonstantinou et al. 2022), and \CRANpkg{nmathresh} (David M. Phillippo et al. 2018) which is
  mainly Bayesian-driven but can be naturally applied to the frequentist framework);
\item
  \emph{the assumed distribution of the input data}: exact distribution, known as \textbf{one-stage}
  \textbf{approach} (typical in the Bayesian framework), or normality approximation, known as
  \textbf{two-stage approach} (typical in the frequentist framework);
\item
  \emph{the modeling approach}: \textbf{arm-based} (\CRANpkg{NMADiagT} (Lu et al. 2020), and
  \CRANpkg{pcnetmeta} (Lin et al. 2017)), or \textbf{contrast-based} (the remaining R packages);
\item
  \emph{the scope breadth}: \textbf{address part of the NMA framework} (\CRANpkg{metapack} (Lim et al. 2022),
  \CRANpkg{NMADiagT} (Lu et al. 2020), \CRANpkg{NMAoutlier} (Petropoulou et al. 2027),
  \CRANpkg{nmaplateplot} (Z. Wang et al. 2021), \CRANpkg{nmarank} (Nikolakopoulou, Schwarzer, and Papakonstantinou 2021), and \CRANpkg{nmathresh}
  (David M. Phillippo et al. 2018)) or \textbf{conduct NMA and assess heterogeneity and inconsistency}
  (the remaining R packages);
\item
  \emph{the outcome structure}: \textbf{mixture of aggregate and individual patient data}
  (\CRANpkg{multinma} (David M. Phillippo 2022)) or \textbf{aggregate data} only (the remaining R
  packages)); and
\item
  \emph{the outcome data type}: \textbf{binary}, \textbf{continuous}, \textbf{multinomial} (\CRANpkg{bnma} (Seo and Schmid 2022) only),
  and so on.
  See Table \ref{tab:table-one}.
\end{itemize}

Most packages fall into many categories. For instance, \CRANpkg{gemtc}
(van Valkenhoef and Kuiper 2021), probably the most popular R package for Bayesian NMA, allows both for
one-stage and two-stage approaches using contrast-based modeling, has a wide scope,
and deals with aggregate outcome data of many types. \CRANpkg{netmeta} (Rücker et al. 2022)
is currently the only R package developed exclusively for NMA in the frequentist
framework based on the graph theory (Ruecker 2012), allows only for a two-stage
approach (contrast-based modeling), has also a wide scope, and accommodates
binary, rates, and continuous aggregate outcome data. On the other side, R packages,
such as \CRANpkg{nmathresh} (David M. Phillippo et al. 2018), \CRANpkg{nmaplateplot} (Z. Wang et al. 2021),
and \CRANpkg{nmarank} (Nikolakopoulou, Schwarzer, and Papakonstantinou 2021) do not perform NMA, but use the NMA results (obtained
using other R packages or statistical software tools) as an input to provide, for
instance, decision-invariant bias-adjustment thresholds and intervals (\CRANpkg{nmathresh}
(David M. Phillippo et al. 2018)), various league tables in heatplot style with all intervention
comparisons (\CRANpkg{nmaplateplot} (Z. Wang et al. 2021)), or an intervention hierarchy
approach tailored to the research question (\CRANpkg{nmarank} (Nikolakopoulou, Schwarzer, and Papakonstantinou 2021)).

Due to the complexity and the wide scope of NMA, the researchers are faced with
a large volume of results, necessary to understand the evidence base, assess the
underlying assumptions, evaluate the quality of the estimated parameters (model
diagnostics), and properly answer the research question, for instance, concerning
the comparative effectiveness of the competing interventions and their hierarchy.
To address the challenges associated with the best reporting of NMA results, the
PRISMA-NMA statement (Hutton et al. 2015) was developed expanding on the PRISMA statement
for pairwise meta-analysis (Page et al. 2021) to provide an extensive checklist with the
essential items pertaining to the NMA results, ensuring completeness in the reporting
of systematic reviews with multiple interventions. The R packages
\CRANpkg{PRISMAstatement} (Wasey 2019) and \CRANpkg{metagear} (Lajeunesse 2021)
facilitate the creation of the PRISMA flow chart and the process of article screening
and data extraction, conforming to the PRISMA statement (Page et al. 2021), and are also
relevant for systematic reviews with multiple interventions. The additional items
in the PRISMA-NMA statement that apply to the NMA framework, such as presentation
and summary of network geometry, inconsistency assessment, league tables and
presentation of intervention hierarchy, are addressed in most R packages either
in a targeted manner (e.g., \CRANpkg{nmaplateplot} (Z. Wang et al. 2021), and
\CRANpkg{nmarank} (Nikolakopoulou, Schwarzer, and Papakonstantinou 2021)) or collectively (\CRANpkg{bnma} (Seo and Schmid 2022),
\CRANpkg{netmeta} (Rücker et al. 2022), \CRANpkg{gemtc} (van Valkenhoef and Kuiper 2021), and \CRANpkg{pcnetmeta}
(Lin et al. 2017)).

Most methodological studies on and systematic reviews with NMA have implemented
Bayesian methods (Efthimiou et al. 2016; Petropoulou et al. 2017). The advantages of the Bayesian
framework (e.g., flexible modeling, allowance of uncertainty in all model parameters,
incorporation of external relevant information and facilitation of probabilistic
statements) (Sutton and Abrams 2001), in conjunction with the dominance of the BUGS software
(Lunn et al. 2009) during the springtime of the NMA framework, may be attributed to the
popularity of Bayesian NMA. The rise of R packages on Bayesian NMA also demonstrates
the acclaim of Bayesian methods from the evidence synthesis community. The rest of
the section pertains to R packages on Bayesian NMA published in the \textbf{CRAN Task View}
\textbf{`Meta-Analysis'} (Dewey and Viechtbauer 2022) that feature a wide methodological and reporting
scope: \CRANpkg{bnma} (Seo and Schmid 2022), \CRANpkg{gemtc} (van Valkenhoef and Kuiper 2021), \CRANpkg{pcnetmeta}
(Lin et al. 2017), and \CRANpkg{rnmamod} (Spineli 2022) (a recent novel contribution).

The R packages \CRANpkg{bnma} (Seo and Schmid 2022), \CRANpkg{gemtc} (van Valkenhoef and Kuiper 2021), and \CRANpkg{pcnetmeta}
(Lin et al. 2017) conduct hierarchical NMA using Markov chain Monte Carlo methods
through the JAGS program (Plummer 2003). However, these packages differ in their
methodological and reporting breadth to some extent: \CRANpkg{bnma} (Seo and Schmid 2022) and
\CRANpkg{gemtc} (van Valkenhoef and Kuiper 2021) have a greater common basis on methods and outputs than
\CRANpkg{pcnetmeta} (Lin et al. 2017). This may be attributed to using the contrast-based
modeling approach, which is the established approach to meta-analysis (trial-specific
relative treatment effects, such as log odds ratio, are pooled across the trials),
whilst \CRANpkg{pcnetmeta} (Lin et al. 2017) considers the arm-based modeling approach,
which deviates from the standard meta-analysis practice (Dias and Ades 2016) and is less
widespread. Currently, the package \CRANpkg{pcnetmeta} (Lin et al. 2017) does not contain
any function to conduct inconsistency evaluation and meta-regression, is limited
only to rankograms in terms of hierarchy measures (Salanti et al. 2022), and considers only
the trace plots as a visual diagnostic tool. On the contrary, \CRANpkg{bnma} (Seo and Schmid 2022)
and \CRANpkg{gemtc} (van Valkenhoef and Kuiper 2021) offer at least one method for inconsistency evaluation,
allow conducting meta-regression and consider a wider variety of hierarchy measures
and diagnostic tools. However, all three R packages provide a small-sized toolkit
with functions regarding the presentation of the relative treatment effects: a league
table for one outcome that appears only in the console, and a forest-plot or table
on the relative treatment effects of all comparisons with the selected intervention.

Furthermore, other common features of the R packages \CRANpkg{bnma} (Seo and Schmid 2022),
\CRANpkg{gemtc} (van Valkenhoef and Kuiper 2021), and \CRANpkg{pcnetmeta} (Lin et al. 2017) include: (i) the
undue reliance on the function \texttt{print()} (the results appear in the console) than
visualisation, and (ii) presenting the results mostly in isolation, restricting
the ability to gain further insights into the performance of the NMA models and
contextualise the results in the light of the strengths and limitations in the
analysis. The package

\begin{table}

\caption{\label{tab:table-one}Features of R packages for network meta-analyses (CRAN Task View)}
\centering
\fontsize{7}{9}\selectfont
\begin{tabular}[t]{l|c|c|c|c|c|c|c|c}
\hline
\multicolumn{1}{c|}{\textbf{Package}} & \multicolumn{2}{c|}{\textbf{Analysis}} & \multicolumn{2}{c|}{\textbf{Modeling approach}} & \multicolumn{2}{c|}{\textbf{Scope breadth}} & \multicolumn{2}{c}{\textbf{Outcome structure}} \\
\cline{1-1} \cline{2-3} \cline{4-5} \cline{6-7} \cline{8-9}
\textbf{} & \textbf{Bayesian} & \textbf{Frequentist} & \textbf{Contrast} & \textbf{Arm} & \textbf{Wide} & \textbf{Narrow} & \textbf{AD \& IPD} & \textbf{AD}\\
\hline
bnma & X &  & X &  & X &  & X & \\
\hline
gemtc & X &  & X &  & X &  & X & \\
\hline
metapack & X &  & X &  &  & X & X & \\
\hline
multinma\textbackslash{}textsuperscript\{1\} & X &  & X &  & X &  &  & X\\
\hline
netmeta &  & X & X &  & X &  & X & \\
\hline
NMADiagT & X &  &  & X &  & X & X & \\
\hline
nmaINLA\textbackslash{}textsuperscript\{2\} & X &  & X &  & X &  & X & \\
\hline
NMAoutlier &  & X & X &  &  & X & X & \\
\hline
nmaplateplot & X & X & X &  &  & X & X & \\
\hline
nmarank\textbackslash{}textsuperscript\{3\} & X & X & X &  &  & X & X & \\
\hline
nmathresh\textbackslash{}textsuperscript\{4\} & X & X & X &  &  & X & X & \\
\hline
pcnetmeta & X &  &  & X & X &  & X & \\
\hline
\multicolumn{9}{l}{\rule{0pt}{1em}\textit{Note: }}\\
\multicolumn{9}{l}{\rule{0pt}{1em}AD, aggregate data; IPD, individual patient data.}\\
\multicolumn{9}{l}{\rule{0pt}{1em}\textsuperscript{1} Estimation using the probabilistic programming language Stan.}\\
\multicolumn{9}{l}{\rule{0pt}{1em}\textsuperscript{2} Use of integrated nested Laplace approximation.}\\
\multicolumn{9}{l}{\rule{0pt}{1em}\textsuperscript{3} Mainly frequentist-driven but can be easily applied to Bayesian results.}\\
\multicolumn{9}{l}{\rule{0pt}{1em}\textsuperscript{4} Mainly Bayesian-driven but can be naturally applied to the frequentist framework.}\\
\end{tabular}
\end{table}

rnmamod provides `dynamic' graphs by combining plots.

\hypertarget{background}{%
\section{Background}\label{background}}

Some packages on interactive graphics include \CRANpkg{plotly} (Sievert 2020) that interfaces with Javascript for web-based interactive graphics, \CRANpkg{crosstalk} (Cheng and Sievert 2021) that specializes cross-linking elements across individual graphics. The recent R Journal paper \CRANpkg{tsibbletalk} (E. Wang and Cook 2021) provides a good example of including interactive graphics into an article for the journal. It has both a set of linked plots, and also an animated gif example, illustrating linking between time series plots and feature summaries.

\hypertarget{customizing-tooltip-design-with}{%
\section{\texorpdfstring{Customizing tooltip design with \pkg{ToOoOlTiPs}}{Customizing tooltip design with }}\label{customizing-tooltip-design-with}}

\pkg{ToOoOlTiPs} is a packages for customizing tooltips in interactive graphics, it features these possibilities.

\hypertarget{a-gallery-of-tooltips-examples}{%
\section{A gallery of tooltips examples}\label{a-gallery-of-tooltips-examples}}

The \CRANpkg{palmerpenguins} data (Horst, Hill, and Gorman 2020) features three penguin species which has a lovely illustration by Alison Horst in Figure \ref{fig:penguins-alison}.

\begin{figure}
\includegraphics[width=1\linewidth,height=0.3\textheight]{penguins} \caption{Artwork by \@allison\_horst}\label{fig:penguins-alison}
\end{figure}

Table \ref{tab:penguins-tab-static} prints at the first few rows of the \texttt{penguins} data:

\begin{table}

\caption{\label{tab:penguins-tab-static}A basic table}
\centering
\fontsize{7}{9}\selectfont
\begin{tabular}[t]{l|l|r|r|r|r|l|r}
\hline
species & island & bill\_length\_mm & bill\_depth\_mm & flipper\_length\_mm & body\_mass\_g & sex & year\\
\hline
Adelie & Torgersen & 39.1 & 18.7 & 181 & 3750 & male & 2007\\
\hline
Adelie & Torgersen & 39.5 & 17.4 & 186 & 3800 & female & 2007\\
\hline
Adelie & Torgersen & 40.3 & 18.0 & 195 & 3250 & female & 2007\\
\hline
Adelie & Torgersen & NA & NA & NA & NA & NA & 2007\\
\hline
Adelie & Torgersen & 36.7 & 19.3 & 193 & 3450 & female & 2007\\
\hline
Adelie & Torgersen & 39.3 & 20.6 & 190 & 3650 & male & 2007\\
\hline
\end{tabular}
\end{table}

Figure \ref{fig:penguins-ggplot} shows an plot of the penguins data, made using the \CRANpkg{ggplot2} package.

\begin{verbatim}
penguins %>% 
  ggplot(aes(x = bill_depth_mm, y = bill_length_mm, 
             color = species)) + 
  geom_point()
\end{verbatim}

\begin{figure}
\centering
\includegraphics{rnmamod-article_files/figure-latex/penguins-ggplot-1.pdf}
\caption{\label{fig:penguins-ggplot}A basic non-interactive plot made with the ggplot2 package on palmer penguin data. Three species of penguins are plotted with bill depth on the x-axis and bill length on the y-axis. Visit the online article to access the interactive version made with the plotly package.}
\end{figure}

\hypertarget{summary}{%
\section{Summary}\label{summary}}

We have displayed various tooltips that are available in the package \pkg{ToOoOlTiPs}.

\hypertarget{references}{%
\section*{References}\label{references}}
\addcontentsline{toc}{section}{References}

\hypertarget{refs}{}
\begin{CSLReferences}{1}{0}
\leavevmode\vadjust pre{\hypertarget{ref-Bucher1997}{}}%
Bucher, H C, G H Guyatt, L E Griffith, and S D Walter. 1997. {``The Results of Direct and Indirect Treatment Comparisons in Meta-Analysis of Randomized Controlled Trials.''} \emph{J Clin Epidemiol} 50 (6): 683--91. \url{https://doi.org/10.1016/s0895-4356(97)00049-8}.

\leavevmode\vadjust pre{\hypertarget{ref-Caldwell2014}{}}%
Caldwell, Deborah M. 2014. {``An Overview of Conducting Systematic Reviews with Network Meta-Analysis.''} \emph{Syst Rev} 3: 109. \url{https://doi.org/10.1186/2046-4053-3-109}.

\leavevmode\vadjust pre{\hypertarget{ref-crosstalk}{}}%
Cheng, Joe, and Carson Sievert. 2021. \emph{{crosstalk}: Inter-Widget Interactivity for HTML Widgets}. \url{https://CRAN.R-project.org/package=crosstalk}.

\leavevmode\vadjust pre{\hypertarget{ref-CRANTaskReview}{}}%
Dewey, Michael, and Wolfgang Viechtbauer. 2022. \emph{{CRAN Task View}: Meta-Analysis}. \url{https://CRAN.R-project.org/view=MetaAnalysis}.

\leavevmode\vadjust pre{\hypertarget{ref-Dias2016}{}}%
Dias, Sofia, and A E Ades. 2016. {``Absolute or Relative Effects? Arm-Based Synthesis of Trial Data.''} \emph{Res Synth Methods} 7 (1): 23--28. \url{https://doi.org/10.1002/jrsm.1184}.

\leavevmode\vadjust pre{\hypertarget{ref-GetRealNMA}{}}%
Efthimiou, Orestis, Thomas P. A. Debray, Gert van Valkenhoef, Sven Trelle, Klea Panayidou, Karel G. M. Moons, Johannes B. Reitsma, Aijing Shang, Georgia Salanti, and GetReal Methods Review Group. 2016. {``GetReal in Network Meta-Analysis: A Review of the Methodology.''} \emph{Res Synth Methods} 7 (3): 236--63. \url{https://doi.org/10.1002/jrsm.1195}.

\leavevmode\vadjust pre{\hypertarget{ref-nmaINLA}{}}%
Guenhan, Burak Kuersad. 2021. \emph{nmaINLA: Network Meta-Analysis Using Integrated Nested Laplace Approximations}. \url{https://CRAN.R-project.org/package=nmaINLA}.

\leavevmode\vadjust pre{\hypertarget{ref-Higgins1996}{}}%
Higgins, J P, and A Whitehead. 1996. {``Borrowing Strength from External Trials in a Meta-Analysis.''} \emph{Stat Med} 15 (24): 2733--49.

\leavevmode\vadjust pre{\hypertarget{ref-palmerpenguins}{}}%
Horst, Allison Marie, Alison Presmanes Hill, and Kristen B Gorman. 2020. \emph{{palmerpenguins}: Palmer Archipelago (Antarctica) Penguin Data}. \url{https://allisonhorst.github.io/palmerpenguins/}.

\leavevmode\vadjust pre{\hypertarget{ref-Hutton2015}{}}%
Hutton, Brian, Georgia Salanti, Deborah M Caldwell, Anna Chaimani, Christopher H Schmid, Chris Cameron, John P A Ioannidis, et al. 2015. {``The PRISMA Extension Statement for Reporting of Systematic Reviews Incorporating Network Meta-Analyses of Health Care Interventions: Checklist and Explanations.''} \emph{Ann Intern Med} 162 (11): 777--84. \url{https://doi.org/10.7326/M14-2385}.

\leavevmode\vadjust pre{\hypertarget{ref-metagear}{}}%
Lajeunesse, Marc J. 2021. \emph{Metagear: Comprehensive Research Synthesis Tools for Systematic Reviews and Meta-Analysis}. \url{https://CRAN.R-project.org/package=metagear}.

\leavevmode\vadjust pre{\hypertarget{ref-Lee2022}{}}%
Lee, Andrew. 2022. {``The Development of Network Meta-Analysis.''} \emph{J R Soc Med} 115 (8): 313--21. \url{https://doi.org/10.1177/01410768221113196}.

\leavevmode\vadjust pre{\hypertarget{ref-metapack}{}}%
Lim, Daeyoung, Ming-Hui Chen, Sungduk Kim, Joseph Ibrahim, Arvind Shah, and Jianxin Lin. 2022. \emph{Metapack: Bayesian Meta-Analysis and Network Meta-Analysis}. \url{https://CRAN.R-project.org/package=metapack}.

\leavevmode\vadjust pre{\hypertarget{ref-pcnetmeta}{}}%
Lin, Lifeng, Jing Zhang, James S. Hodges, and Haitao Chu. 2017. {``Performing Arm-Based Network Meta-Analysis in {R} with the {pcnetmeta} Package.''} \emph{Journal of Statistical Software} 80 (5): 1--25. \url{https://doi.org/10.18637/jss.v080.i05}.

\leavevmode\vadjust pre{\hypertarget{ref-NMADiagT}{}}%
Lu, Boyang, Qinshu Lian, James S. Hodges, Yong Chen, and Haitao Chu. 2020. \emph{NMADiagT: Network Meta-Analysis of Multiple Diagnostic Tests}. \url{https://CRAN.R-project.org/package=NMADiagT}.

\leavevmode\vadjust pre{\hypertarget{ref-Lunn2009}{}}%
Lunn, David, David Spiegelhalter, Andrew Thomas, and Nicky Best. 2009. {``The BUGS Project: Evolution, Critique and Future Directions.''} \emph{Stat Med} 28 (25): 3049--67. \url{https://doi.org/10.1002/sim.3680}.

\leavevmode\vadjust pre{\hypertarget{ref-nmarank}{}}%
Nikolakopoulou, Adriani, Guido Schwarzer, and Theodoros Papakonstantinou. 2021. \emph{Nmarank: Complex Hierarchy Questions in Network Meta-Analysis}. \url{https://CRAN.R-project.org/package=nmarank}.

\leavevmode\vadjust pre{\hypertarget{ref-Page2020}{}}%
Page, Matthew J, Joanne E McKenzie, Patrick M Bossuyt, Isabelle Boutron, Tammy C Hoffmann, Cynthia D Mulrow, Larissa Shamseer, et al. 2021. {``The PRISMA 2020 Statement: An Updated Guideline for Reporting Systematic Reviews.''} \emph{BMJ} 372: n71. \url{https://doi.org/10.1136/bmj.n71}.

\leavevmode\vadjust pre{\hypertarget{ref-Papakonstantinou2022}{}}%
Papakonstantinou, Theodoros, Georgia Salanti, Dimitris Mavridis, Gerta Rücker, Guido Schwarzer, and Adriani Nikolakopoulou. 2022. {``Answering Complex Hierarchy Questions in Network Meta-Analysis.''} \emph{BMC Med Res Methodol} 22 (1): 47. \url{https://doi.org/10.1186/s12874-021-01488-3}.

\leavevmode\vadjust pre{\hypertarget{ref-Petropoulou2017}{}}%
Petropoulou, Maria, Adriani Nikolakopoulou, Areti-Angeliki Veroniki, Patricia Rios, Afshin Vafaei, Wasifa Zarin, Myrsini Giannatsi, et al. 2017. {``Bibliographic Study Showed Improving Statistical Methodology of Network Meta-Analyses Published Between 1999 and 2015.''} \emph{J Clin Epidemiol} 82: 20--28. \url{https://doi.org/10.1016/j.jclinepi.2016.11.002}.

\leavevmode\vadjust pre{\hypertarget{ref-NMAoutlier}{}}%
Petropoulou, Maria, Guido Schwarzer, Agapios Panos, and Dimitris Mavridis. 2027. \emph{NMAoutlier: Detecting Outliers in Network Meta-Analysis}. \url{https://CRAN.R-project.org/package=NMAoutlier}.

\leavevmode\vadjust pre{\hypertarget{ref-multinma}{}}%
Phillippo, David M. 2022. \emph{Multinma: Bayesian Network Meta-Analysis of Individual and Aggregate Data}. \url{https://CRAN.R-project.org/package=multinma}.

\leavevmode\vadjust pre{\hypertarget{ref-nmathresh}{}}%
Phillippo, David M, Sofia Dias, A E Ades, Vanessa Didelez, and Nicky J Welton. 2018. {``Sensitivity of Treatment Recommendations to Bias in Network Meta-Analysis.''} \emph{Journal of the Royal Statistical Society: Series A (Statistics in Society)} 181 (3): 843--67. \url{https://doi.org/10.1111/rssa.12341}.

\leavevmode\vadjust pre{\hypertarget{ref-Plummer2003}{}}%
Plummer, Martyn. 2003. \emph{JAGS: A Program for Analysis of Bayesian Graphical Models Using Gibbs Sampling}. Edited by Kurt Hornik, Friedrich Leisch, and Achim Zeileis. Technische Universität Wien, Vienna, Austria. \url{https://www.R-project.org/conferences/DSC-2003/Proceedings/Plummer.pdf}.

\leavevmode\vadjust pre{\hypertarget{ref-R2022}{}}%
R Core Team. 2022. \emph{{R: A Language and Environment for Statistical Computing}. Foundation for Statistical Computing, Vienna, Austria}. \url{https://www.R-project.org/}.

\leavevmode\vadjust pre{\hypertarget{ref-netmeta}{}}%
Rücker, Gerta, Ulrike Krahn, Jochem König, Orestis Efthimiou, Annabel Davies, Theodoros Papakonstantinou, and Guido Schwarzer. 2022. \emph{Netmeta: Network Meta-Analysis Using Frequentist Methods}. \url{https://CRAN.R-project.org/package=netmeta}.

\leavevmode\vadjust pre{\hypertarget{ref-Ruecker2012}{}}%
Ruecker, Gerta. 2012. {``Network Meta-Analysis, Electrical Networks and Graph Theory.''} \emph{Res Synth Methods} 3 (4): 312--24. \url{https://doi.org/10.1002/jrsm.1058}.

\leavevmode\vadjust pre{\hypertarget{ref-EvidenceBasedMedicine}{}}%
Sackett, David L, William M Rosenberg, J A Gray, R B Haynes, and W S Richardson. 1996. {``Evidence Based Medicine: What It Is and What It Isn't.''} \emph{BMJ} 312 (7023): 71--72. \url{https://doi.org/10.1136/bmj.312.7023.71}.

\leavevmode\vadjust pre{\hypertarget{ref-Salanti2012}{}}%
Salanti, Georgia. 2012. {``Indirect and Mixed-Treatment Comparison, Network, or Multiple-Treatments Meta-Analysis: Many Names, Many Benefits, Many Concerns for the Next Generation Evidence Synthesis Tool.''} \emph{Res Synth Methods} 3 (2): 80--97. \url{https://doi.org/10.1002/jrsm.1037}.

\leavevmode\vadjust pre{\hypertarget{ref-Salanti2022}{}}%
Salanti, Georgia, Adriani Nikolakopoulou, Orestis Efthimiou, Dimitris Mavridis, Matthias Egger, and Ian R White. 2022. {``Introducing the Treatment Hierarchy Question in Network Meta-Analysis.''} \emph{Am J Epidemiol} 191 (5): 930--38. \url{https://doi.org/10.1093/aje/kwab278}.

\leavevmode\vadjust pre{\hypertarget{ref-SAS2020}{}}%
SAS Institute. 2020. \emph{{The SAS System for Windows}. Release 9.4. Cary, NC: SAS Inst}. \url{https://www.sas.com}.

\leavevmode\vadjust pre{\hypertarget{ref-bnma}{}}%
Seo, Michael, and Christopher Schmid. 2022. \emph{Bnma: Bayesian Network Meta-Analysis Using 'JAGS'}. \url{https://CRAN.R-project.org/package=bnma}.

\leavevmode\vadjust pre{\hypertarget{ref-plotly}{}}%
Sievert, Carson. 2020. \emph{{Interactive Web-Based Data Visualizatio}n with r, Plotly, and Shiny}. Chapman; Hall/CRC. \url{https://plotly-r.com}.

\leavevmode\vadjust pre{\hypertarget{ref-rnmamod}{}}%
Spineli, Loukia M. 2022. \emph{Rnmamod: Bayesian Network Meta-Analysis with Missing Participants}. \url{https://CRAN.R-project.org/package=rnmamod}.

\leavevmode\vadjust pre{\hypertarget{ref-Stata}{}}%
StataCorp. 2021. \emph{{Stata Statistical Software: Release 17}. College Station, TX: StataCorp LLC}. \url{http://www.stata.com}.

\leavevmode\vadjust pre{\hypertarget{ref-Sutton2001}{}}%
Sutton, Alex J, and Keith R Abrams. 2001. {``Bayesian Methods in Meta-Analysis and Evidence Synthesis.''} \emph{Stat Methods Med Res} 10 (4): 277--303. \url{https://doi.org/10.1177/096228020101000404}.

\leavevmode\vadjust pre{\hypertarget{ref-gemtc}{}}%
van Valkenhoef, Gert, and Joel Kuiper. 2021. \emph{Gemtc: Network Meta-Analysis Using Bayesian Methods}. \url{https://CRAN.R-project.org/package=gemtc}.

\leavevmode\vadjust pre{\hypertarget{ref-RJ-2021-050}{}}%
Wang, Earo, and Dianne Cook. 2021. {``Conversations in Time: Interactive Visualisation to Explore Structured Temporal Data.''} \emph{The R Journal}. \url{https://doi.org/10.32614/RJ-2021-050}.

\leavevmode\vadjust pre{\hypertarget{ref-nmaplateplot}{}}%
Wang, Zhenxun, Lifeng Lin, Shanshan Zhao, and Haitao Chu. 2021. \emph{Nmaplateplot: The Plate Plot for Network Meta-Analysis Results}. \url{https://CRAN.R-project.org/package=nmaplateplot}.

\leavevmode\vadjust pre{\hypertarget{ref-PRISMAstatement}{}}%
Wasey, Jack O. 2019. \emph{PRISMAstatement: Plot Flow Charts According to the "PRISMA" Statement}. \url{https://CRAN.R-project.org/package=PRISMAstatement}.

\end{CSLReferences}

\bibliography{RJreferences.bib}

\address{%
Loukia M. Spineli\\
Midwifery Research and Education Unit\\%
Hannover Medical School\\ Carl-Neuber-Strasse 1, 30625, Hannover, Germany\\
%
\url{https://www.github.com/LoukiaSpin}\\%
\textit{ORCiD: \href{https://orcid.org/0000-0001-9515-582X}{0000-0001-9515-582X}}\\%
\href{mailto:Spineli.Loukia@mh-hannover.de}{\nolinkurl{Spineli.Loukia@mh-hannover.de}}%
}

\address{%
Chrysostomos Kalyvas\\
Biostatistics and Research Decision Sciences\\%
MSD Europe Inc., Brussels, Belgium\\
%
\url{https://www.github.com/ckalyvas}\\%
\textit{ORCiD: \href{https://orcid.org/0000-0003-0606-4518}{0000-0003-0606-4518}}\\%
\href{mailto:chrysostomos.kalyvas@merck.com}{\nolinkurl{chrysostomos.kalyvas@merck.com}}%
}

\address{%
Katerina Papadimitropoulou\\
Health Economics and Market Access\\%
Amaris Consulting, Lyon, France\\
%
\url{https://www.github.com/Katerina-Pap}\\%
\textit{ORCiD: \href{https://orcid.org/0000-0002-5732-4044}{0000-0002-5732-4044}}\\%
\href{mailto:katerina.papadimitropoulou@gmail.com}{\nolinkurl{katerina.papadimitropoulou@gmail.com}}%
}
